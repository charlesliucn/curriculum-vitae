\resheading{项目经历}
  \begin{itemize}[leftmargin=*]
    \item \ressubsingleline{家庭场景多通道中文语音识别系统}{系统}{2020.01 -- 2020.03}
      {\small
      \begin{itemize}
        \item 项目简介: 个人参加MagicSpeechNet语音识别竞赛,对于多通道的家庭场景语音,采用去混响、多通道语音融合方法及数据增强策略,构建中英文发音词典及语言模型,采用重打分提升识别效果。
        \item 个人工作: 多通道语音的前端处理:先用WPE方法进行多通道混响去除,再使用BeamformIt波束形成得到单通道语音。从非语音区段提取噪声信息进行语音增强,并使用音量扰动、速度扰动进行数据扩充。中文发音词典基于开源的DaCiDian,并建立中英文音素映射关系构建英文常用词发音词典。声学模型使用TDNN-F模型,语言模型使用最大熵语言模型,并用双层Bi-LSTM语言模型重打分。
      \end{itemize}
      }
    \item \ressubsingleline{语音识别后处理系统研究与设计}{后处理}{2019.09 -- 2020.02}
      {\small
      \begin{itemize}
        \item 项目简介: 从识别结果的可读性、准确性、规范化和实时性四个角度,研究设计语音识别后处理系统。
        \item 个人工作: 基于LSTM-CRF序列标注模型,添加解码的语音停顿信息,对识别文本进行句子分割并恢复标点。使用OpenGRM语法工具,设计中文文本逆正则化转换规则。对于文本纠错,使用N元语言模型和LSTM语言模型打分后,基于异常点检测检出错误,并采用候选词替换和打分策略纠正错误。
      \end{itemize}
      }
    \item \ressubsingleline{基于x-vector的说话人适应训练}{声学模型}{2019.08 -- 2019.10}
      {\small
      \begin{itemize}
        \item 项目简介: 将说话人识别的x-vector用于声学模型的说话人适应,得到比用i-vector更优的识别结果。
        \item 个人工作: 使用TED-LIUM语料,根据说话人标注训练TDNN模型,提取x-vector说话人特征。在语音识别场景下对x-vector提取进行优化:训练数据筛选;细分说话人标签以增加说话人丰富性;去除LDA降维模块,直接从网络提取所需维度特征。将x-vector与MFCC拼接进行说话人适应训练。
      \end{itemize}
      }
    \item \ressubsingleline{低资源条件下关键词检测评测}{关键词检测}{2019.03 -- 2019.06}
      {\small
      \begin{itemize}
        \item 项目简介: 基于NIST的OpenSAT 2019公开评测(语音识别和关键词检测任务),提升低资源(普什图语)语音识别和关键词检测的性能指标,重点采取多种策略提升集外关键词的检测效果。
        \item 个人工作:
        将前沿的跳帧TDNN-F声学模型应用于关键词检测任务。对于集外关键词,一方面利用集外数据筛选高频词加入词表,另外利用morfessor基于词素语言模型生成潜在新词,降低模型的集外词比例。采用代理词策略,提升集外词的检测性能。通过ROVER算法融合多个系统,得到最优结果。
      \end{itemize}
      }
    \item \ressubsingleline{电信电话场景商用语音识别系统}{系统}{2018.05 -- 2019.01}
     {\small
      \begin{itemize}
        \item 项目简介: 搭建电信电话场景商用语音识别系统,训练使用三万小时级语音数据及400G通用语料库。
        \item 个人工作:
        采用SRILM工具并行化统计词频,训练N-gram通用语言模型。针对应用场景,对发音词典扩充人名和地名实体词,并提高人名、地名等实体词的各阶频次。通用语言模型与场景语言模型进行裁剪与插值提高鲁棒性。三万小时语音为分批数据,采用增量学习方法快速训练和更新通用声学模型。
      \end{itemize}
      }
    \item \ressubsingleline{低资源条件语言模型建模优化}{语言模型}{2017.11 -- 2018.06}
      {\small
      \begin{itemize}
        \item 项目简介: 基于NIST的 OpenSAT 2017评测的语音识别任务,采取g2p方法扩充发音词典,并根据文本相似度进行数据筛选,将预训练词向量用于LSTM语言模型,提升低资源场景的语言建模能力。
        \item 个人工作: 对网络文本进行预处理,用高频词扩充词表,降低集外词比例;使用g2p方法生成集外词发音词典。筛选出与训练数据语义相近的文本作为扩充数据,方法包括:困惑度排序、交叉熵距离排序、TF-IDF文本相似度、基于doc2vec的文本相似度等。对比Skip-Gram、CBOW、Glove三种词向量预训练方法,用于LSTM语言模型的权重初始化。优化后的语言模型帮助词错误率有3.9\%的绝对降低。
      \end{itemize}
      }
   \end{itemize}