\resheading{实习经历}
  \begin{itemize}[leftmargin=*]
  \item \ressubsingleline{天津华控智达科技有限公司}{语音识别算法实习生}{2019.06 -- 2020.01}

    \begin{itemize}[leftmargin=*]
      \small{
      \item \textbf{声学模型增量学习:} 对于新语音数据,利用原始声学模型对齐(长音频需切分)后,在原模型基础上以较小学习率训练。采取迁移学习的思路,保持原模型各层具有较小学习率,着重对增加新神经网络层训练。
      \item \textbf{解码图构建流程优化:} 对HCLG解码图的构图方法加以更改,采用HCL+G的合成方法,以应对语言模型更新频繁的问题,避免重复构建HCL部分的WFST。
      \item \textbf{语音识别后处理研究:} 对语音识别系统的错误情况进行诊断分析,添加文本纠错模块,并使用序列标注方法恢复标点符号,基于OpenGRM工具设计规则进行文本逆正则化。
       }
    \end{itemize}
  

  \item \ressubsingleline{北京华控智加科技有限公司}{语音识别算法实习生}{2018.09 -- 2019.05}

    \begin{itemize}[leftmargin=*]
      \small{
      \item \textbf{语言模型文本清洗:} 针对400G量级的文本数据,使用正则表达式和数据流处理命令初步清洗数据,并基于相对交叉熵过滤低相关度的文本,采用多线程进行文本正则化与分词。
      \item \textbf{训练集长语音切分:} 对于长音频,对比使用两种切分方法:(1) 对齐法:将长音频与标注做强制对齐,在静音处进行切分;(2) 解码法:将解码结果与标注数据匹配对应,根据预设的时长阈值进行切分。 
      \item \textbf{中文发音字典构建:} 采用错误率低的g2pC工具获取拼音,基于音素与拼音映射关系,生成发音词典。
      \item \textbf{语音识别系统训练:} 三万小时级声学模型的训练与优化经验。针对400G的通用语料,通用语言模型训练采用并行化统计词频,降低服务器内存压力。日语、闽南语等多个小语种的声学模型训练与优化。
    }
    \end{itemize}

  \end{itemize}