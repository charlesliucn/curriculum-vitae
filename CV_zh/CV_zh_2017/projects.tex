\resheading{项目经历}
  \begin{itemize}[leftmargin=*]
    \item
      \ressubsingleline{基于深度学习的移动大数据挖掘及用户行为建模}{Python}{2017.03至今}
      {\small
      \begin{itemize}
        \item 环境配置: Ubuntu, Python, Tensorflow
        \item 项目简介: 运用传统方法(马尔科夫链等)和深度学习方法(RNN等),对从时间和空间两个维度对用户的移动数据进行建模分析及预测,期望根据用户现有的轨迹数据预测未来某一时刻的位置。
        \item 个人工作: 实现传统算法,并基于Tensorflow等框架尝试对深度学习算法的性能进行复现、评估与改进。 
      \end{itemize}
      }
    \item \ressubsingleline{中文连续语音识别}{Python}{2017.03 -- 2017.05}
    {\small
    \begin{itemize}
      \item 环境配置: Ubuntu,Kaldi,PyQt
      \item 项目简介: 基于Ubuntu平台,安装编译Kaldi环境,分别基于timit(英文)和thchs30(中文)数据集训练得到语音识别模型,并且基于训练得到的模型实现实时语音识别。
      \item 个人工作: 基于thchs30中文数据集,训练得到单音素和三音素语音模型;并且使用PyQt设计了图形用户界面,实现用户点击按键后实时进行语音识别的交互式程序。
    \end{itemize}
    }
    \item
      \ressubsingleline{旧金山湾区公共自行车数据分析与可视化}{R}{2017.01 -- 2017.02}
      {\small
      \begin{itemize}
        \item 环境配置: Windows, R, R Shiny App
        \item 项目简介: 使用旧金山湾区2014至2016年间公共自行车的使用数据,进行可视化分析,并使用基本统计方法发现内在规律。
        \item 个人工作: 从空间和时间两个角度,使用基础的机器学习方法(线性回归、LASOO回归等)分析天气等因素对公共自行车使用情况的影响,并基于分析结果,对公共自行车的分配提出了可行的建议,实现了一款Shiny App,向公共自行车使用者提供了路线查询等功能。
      \end{itemize}
      }
    \item \ressubsingleline{MIPS处理器设计与实现}{Verilog}{2016.06  -- 2016.07}
    {\small
    \begin{itemize}
      \item 环境配置: Windows,Vivado,Modelsim,Basys 3.0开发板
      \item 项目简介: 组建3人小组,使用Verilog编程实现32位MIPS处理器,并实现定时器和UART外设的相关功能。
      \item 个人工作: 基于单周期的模块化原理,独立实现了各单元模块的设计和调试,将各模块综合并加入外设功能,基于Basys 3.0 FPGA开发板实现了单周期处理器,提高了个人硬件设计调试以及团队合作的能力。
    \end{itemize}
    }
    \item
    \ressubsingleline{基于MATLAB的图像处理}{MATLAB}{2016.03 -- 2016.04}
    {\small
    \begin{itemize}
      \item 环境配置: Windows, MATLAB
      \item 项目简介: 对图像进行压缩编码和信息隐藏,进行相关处理、算法实现和比较。
      \item 个人工作: 实现了JPEG压缩编解码算法,并比较了空域和DCT变换域的信息隐藏与提取技术。另外还基于彩色直方图方法实现了在图像中进行人脸检测的功能。
    \end{itemize}
    }
    % \item
    % \ressubsingleline{烟草图像杂质识别}{MATLAB}{2017.03 -- 2017.04}
    % {\small
    % \begin{itemize}
    %   \item 环境配置: Windows,MATLAB
    %   \item 项目简介: 使用统计信号的处理方法,对烟草图像中的杂质进行识别和剔除。
    %   \item 个人工作: 基于烟草和杂质RGB和HSV的分布进行高斯拟合,利用训练数据得到分离效果较好的阈值,在测试数据上杂质查全率达到90\%以上。
    % \end{itemize}
    % }
    % \item \ressubsingleline{自适应滤波算法文献调研与综述}{\LaTeX}{2015.11 -- 2016.02}
    % {\small
    % \begin{itemize}
    %   \item 环境配置: Windows,\LaTeX
    %   \item 项目简介: 阅读中英文自适应滤波算法的相关文献进行文献综述,尝试对算法进行改进。
    %   \item 个人工作: 阅读了大量近些年国内外有关自适应滤波算法的文献,对不同自适应滤波算法的原理与性能进行了学习并总结,英文文献阅读与综述能力得到了显著提高。
    % \end{itemize}
    % }

    % \item \ressubsingleline{站台无线计时器设计}{电路设计}{2016.07 -- 2016.08}
    % {\small
    % \begin{itemize}
    %   \item 环境配置: Windows,Latex
    %   \item 项目简介: 组建2人小组,数字电路与模拟电路结合,实现站台无线计时的功能。
    %   \item 个人工作: 根据功能需求,自主设计了发射机与接收机,并且逐步调试电路参数,对接收到的含有噪声的微弱信号进行处理,通过解调从信号中提取出来,实现了模拟电路模块的基本功能。
    % \end{itemize}
    % }

  \end{itemize}